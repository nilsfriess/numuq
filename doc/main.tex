\documentclass[a4paper,parskip=half]{scrartcl}

\usepackage[utf8]{inputenc}
\usepackage[light]{kpfonts}
\usepackage[english]{babel}
\usepackage{microtype}
\usepackage[hidelinks]{hyperref}

\usepackage[top=2cm,left=2.5cm,bottom=3cm,right=2.5cm]{geometry}

\addtokomafont{disposition}{\rmfamily}

\title{\LARGE Guide to Working with the Research Group's Website}
\date{\normalsize Last updated: \today}

\setcounter{secnumdepth}{0}

\begin{document}%
\maketitle
\tableofcontents
\newpage
\section{Getting started}
The web page is built with \texttt{hugo}, a static site generator that can be
downloaded at \url{https://gohugo.io}. This makes editing the web page very easy
since \texttt{hugo} generates all the html, css and javascript files from simple
markdown files; no html to has to be edited by hand.

Install \texttt{hugo} and \texttt{git} and setup an account with ssh access on
the katana server (contact Joachim Simon). If everything is set up, clone the
repository that contains all the files to your computer by
\begin{center}
  \texttt{git clone \url{URL} numuq}.
\end{center}
This will create a folder \texttt{numuq} that contains all the necessary
files. Navigate into the folder (\texttt{cd numuq}) and run \texttt{hugo} with
the command \texttt{hugo server}. This will start a local web server and open a
local version of the web page in your web browser. If not, open your web browser
and navigate to the url
\begin{center}
  \url{http://localhost:1313}.
\end{center}
You can now start editing the files; the web page will reload
automatically. When you are finished editing the website stop the \texttt{hugo
  server} command and simply run \texttt{hugo} (without arguments). This will
generate all the files that can then be published to the server. The publishing
process is described in detail below. Note that this documentation is also part
of the repository and if you find errors or want to extend it, please feel free
to do so.

\newpage
\section{Recurring tasks}
\texttt{hugo} makes editing and creating pages very easy. It is not necessary to
write html, css or javascript; \texttt{hugo} uses \texttt{markdown} files that
are easier to work with than html files.  This also makes sure that everything
is displayed correctly while you don't have to take care of the layout
yourself. Below is a list of recurring tasks such as adding/ editing lectures or
editing your personal page.

\subsection{Before every task}
Before you change anything, you have to update your local version to match the
one on the server (since someone else could have changed something so that your
local version and the version on the server do not match anymore).

Just navigate to the folder that contains the repository on your computer and
run
\begin{center}
  \texttt{git pull}.
\end{center}
If you do not do this, errors might occur when you try to publish your changes
and the changes are conflicting with the files on the server. You can then start
the development server by running the command \texttt{hugo server}.

\subsection{Publishing your changes/ Uploading the files to the server}
When you are done adding new pages/ changing page content etc., you have to
\emph{commit} and \emph{push} the files to the git repository on the server.
This is done as follows: Stop the development server that you started earlier.
Then, you can create the actual html/css/javascript files that will be uploaded
to the server by simply running the command \texttt{hugo}. If no errors
occurred, you can type \texttt{git status} to see the files \texttt{git} thinks
have changed.  These should be the actual markdown files you created or changed
but also the corresponding html files in the folder \texttt{public/}
(\textbf{This step is very important since otherwise the markdown files and
  corresponding html files do not match and your changes will not get
  published}).

If everything looks okay, type \texttt{git add -u}. This tells \texttt{git} that
you want to track the changes you made to files that already existed in the
repository before you started. If you also created new pages, you have to add
them manually by typing for each new page (again: don't forget the newly created
pages in the \texttt{public} folder)
\begin{center}
  \texttt{git add path/to/new/page}
\end{center}
You can again run \texttt{git status} to see if you have added all the files you
added/modified.

Now you can commit the changes to the repository by typing
\begin{center}
  \texttt{git commit -m "Add new page for course xyz"}
\end{center}
The message after the parameter \texttt{-m} is the so called \emph{commit
  message} and should describe your changes. Now you can upload the files to the
server by running
\begin{center}
  \texttt{git push}
\end{center}
If everything is setup correctly, your files are now uploaded to the server and
will automatically get published to the web. Note that it might take a few
minutes until you see the changes; you might also want to reload the page using
\texttt{Ctrl + Shift + R} (Linux/Windows) or \texttt{Cmd + Shift + R} (Mac) to
force your web browser to fetch the updated version.

\subsection{Adding a new semester}
Adding a new entry on the teaching page is done by editing the file
\begin{center}
  \texttt{content/teaching/\_index.md}\,.
\end{center}
The file is broken up into different semesters, which are called \emph{terms} in
this context. The first term in the file automatically gets interpreted as the
current semester (see the Teaching page on the website to see the difference
between the way the semesters are displayed). The remaining terms are rendered
as ``past terms''. This makes adding new terms very easy. Simply add the new
term above the term that is currently at the top. The syntax that defines a new
term is
\begin{verbatim}
  [[term]]
    title = "WiSe 2019/20"
\end{verbatim}

\subsection{Adding a new lecture/ seminar/ course}
Under each term in the file \texttt{content/teaching/\_index.md}, you can define
as many lectures as you want. Note, that no matter if it is a lecture, a seminar
or something else you want to add, in \texttt{hugo} it is always referred to as
a lecture. Every lecture has the following properties
\begin{itemize}
\item A title/name called \texttt{title}
\item A link to its LSF page called \texttt{lsf}
\item A link to its moodle page called \texttt{moodle}
\item A link to a separate course page called \texttt{page} (setting up a course
  page is described below)
\end{itemize}
Using these properties one can define a lecture using
\begin{verbatim}
  [[term.lecture]]
    title = "Example Lecture"
    lsf = "https:://lsf.uni-heidelberg.de/xyz"
    moodle = "https://moodle.uni-heidelberg.de/xyz"
    page = "examplelecture"
\end{verbatim}
None of the properties (except for the title) is mandatory. If the \texttt{page}
property is set, the lecture is automatically turned into a link that links to a
separate course page. Of course, this page has to be created which is done by
creating a markdown file in the folder \texttt{content/teaching}. The name of
the file has to equal the value of the page parameter. For example, in the
example above we would need to create a file named \texttt{examplelecture.md}.

At the top of this file, enclosed in three plus symbols ``+++'' the title of the
lecture has to be given. This can differ from the title given above so that on
the teaching page a short name can be given (e.g. ``Numerik 0'' on the teaching
page and ``Einführung in die Numerik'' on the detail page). Below the plus
symbols arbitrary content can be added in markdown format. It is easiest to copy
an existing file and edit it according to your needs.

\subsection{Adding a page for a group member}
Adding a new group member is done in two steps. First, we add the a new entry to
the file \texttt{content/people/\_index.md}. There are two types of entries, one
for PhD students and one for postdocs. Both have the following three properties:
\begin{itemize}
\item The name of the member called \texttt{name}.
\item A short description of the research area called \texttt{topic}.
\item The name of the personal page called \texttt{page} (explained below).
\end{itemize}
The second step is to create and setup the personal page. To this end, we first
create a markdown file with the name set to the value of \texttt{page} from
above. This file contains additional information about the respective member
such as contact details and (optionally) a picture. It is easiest to simply copy
an existing page and edit it accordingly. Most of the parameters are
self-explanatory expect for \texttt{pub\_name} and \texttt{research\_interest}.

\paragraph{\texttt{pub\_name}}
Every group member's page contains a link to the publications page. The
publications page allows the user to filter the entries (e.g. by author) and the
value of \texttt{pub\_name} sets this value for the respective group member when
the user clicks on the ``see publications'' link of a member. Thus, this value
should in most cases be set to the last name.

\paragraph{\texttt{research\_interest}}
This parameter contains a more detailed description of the research interest. To
add multi-line content, you can use the following syntax
\begin{verbatim}
  research_interest = """
  I am interested in 

  * Numerical analysis

  * Uncertainty Quantification
  """
\end{verbatim}
This would result in a list, similar to the following:

\noindent I am interested in
\begin{itemize}
\item Numerical Analysis
\item Uncertainty Quantification
\end{itemize}

\subsection{Using \LaTeX}
Generally, everywhere where you can put text, you can also put Latex code.  Of
course, only a subset of the Latex functionality is supported but simple
equations or symbols can be used. Note, however, that due to the fact that
\texttt{hugo} does not (yet) support the library that is responsible for
converting the Latex code to HTML, you have to escape every backslash
``\textbackslash'', i.e. you have to duplicate every backslash. Apart from that,
you can use it as usual. To use display the maths in inline mode you have to
enclose the Latex code by a dollar sign ``\$'', to render an equation in display
mode you have to use two dollar sings ``\$\$''. For instance, to achieve
something like
\begin{equation*}
  \hat{\mathbb{E}}(f) = \frac{1}{N} \sum_{i=1}^N f(X_i)
\end{equation*}
you would type
\begin{verbatim}
  $$
    \\hat{\\mathbb{E}}(f) = \\frac{1}{N} \\sum_{i=1}^N f(X_i)
  $$
\end{verbatim}


% \newpage
% \section{General documentation of the web page}
% \subsection{Folder structure}
% After cloning the repository you will found (among others) the following
% folders:
% \begin{itemize}
% \item content,
% \item data,
% \item layouts,
% \item static.
% \end{itemize}
% The first folder contains -- as the name suggests -- the actual contents. It
% contains a sub folder for every menu entry and in these sub folders the actual
% markdown files.

% The data folder contains data that can be used across different sites.
% Currently, only the descriptions of the research areas that are shown on the
% home page and the research page are stored in files in this directory.

% The layouts folder contains the actual html templates that are used as a basis
% for the various content pages. Changes made here affect all the pages that use
% the according template (e.g. changes made to the file
% \texttt{layouts/teaching/single.html} affect every course page located in the
% \texttt{content/teaching} folder). If you only want to change some contents of
% the page (e.g. add a new course page for a lecture), you don't have to change
% anything in these files.

% The static folder contains the css and javascript files but also contains all
% the images, pdf files, and BibTeX files containing the publications.


% \subsection{The \texttt{content} folder}
% \subsubsection*{The \texttt{content/news/\_index.md} file}
% This file contains a list of events.

% \subsubsection*{The \texttt{content/people} folder}
% The \texttt{\_index.md} file contains a list of group members. List entries have
% the following format
% \begin{verbatim}
% [[role]]
%   name = "John Doe"
%   topic = "Very important research topic"
%   page = "john"
% \end{verbatim}
% The parameter \texttt{role} can be one of the following
% \begin{itemize}
% \item \texttt{phd}: For PhD students
% \item \texttt{postdoc}: For postdoctoral researchers
% \item \texttt{member}: For affiliated members
% \end{itemize}
% The \texttt{name} and \texttt{topic} parameter are self-explanatory.  The page
% parameter is only available for \texttt{phd} and \texttt{postdoc} entries. This
% parameter is \textbf{not} required. If present, there must exist a file in the
% people directory called \texttt{john.md} (for the example above). When the
% parameter is set, the name of the group member is made into a link that points
% to a personal page of that member. The contents of this page are described by
% the \texttt{john.md} file. This file provides the following parameters (the
% parameters in italic are required; the other ones can be omitted):
% \begin{itemize}
% \item \emph{title}: The title shown at the top of the web browser
% \item \emph{name}: The name of the group member. This parameter can be formatted
%   using markdown syntax; to make the last name bold set this to "John **Doe**"
% \item \emph{email}
% \item phone
% \item room
% \item avatar: A string that points to a picture; the path should be relative to
%   the \texttt{static} folder; for example: "img/john.jpeg"
% \item google\_scholar: If present, a link to the Google Scholar profile is added
% \item \emph{pub\_name}: This name is used when clicking on the link to the
%   publications page. This page lists the publications of the whole research
%   group. However, it provides a filter that lets users filter by author. This
%   value gets automatically put into the filter input. Should either be just the
%   last name or last name and the first letter of the first name, seperated by
%   comma, e.g.: either "Doe" or "Doe, J".

% \item \emph{research\_interest}: Research interest of the group member. Markdown
%   syntax can be used to make lists; \LaTeX{} can be used to typeset maths.
% \end{itemize}


% When your are finished editing all the files and want to publish the page you
% also have to \emph{commit} the changes to the GitHub repository.  This is done
% by the following steps:
% \begin{itemize}
% \item If you have created any new files (e.g. new pictures, new PDF files, new
%   course overview pages etc.) you have tell \texttt{git} that this files are
%   also part of the repository. Example: If you have created a new course page
%   named \texttt{Num2.md} inside the folder \texttt{content/teaching} you have
%   to run the command
%   \begin{center}
%     \texttt{git add content/teaching/Num2.md}.
%   \end{center}
%   (Note: this step is only necessary if you have created \emph{new} files).
% \item After you have added all the new files you can commit your changes. This
%   can be done by the command
%   \begin{center}
%     \texttt{git commit -m "Added new teaching page for Num2 course"}.
%   \end{center}
%   The string after the \texttt{-m} flag is called the \emph{commit message}
%   and should describe your changes.
% \item Finally you can upload the changes to the GitHub repository (by now, all
%   the changes are only local on your machine). This can be done with the
%   command
%   \begin{center}
%     \texttt{git push} \quad or \quad \texttt{git push -u origin master}.
%   \end{center}
%   Use the second command if the first one fails. Depending on your setup,
%   \texttt{git} might prompt for your credentials. Type in the email and
%   password you used to create the account on \url{github.com}.
% \end{itemize}
% Now you can upload the files from the public folder to the webserver.
\end{document}

%%% Local Variables:
%%% mode: latex
%%% TeX-master: t
%%% End:
